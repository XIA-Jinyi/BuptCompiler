\documentclass{CompilerAssignment}
\usepackage{multicol}
\usepackage{multirow}
\usepackage{longtable}

\newcommand{\mbfa}{\mathbf{a}}
\newcommand{\bfa}{\textbf{a}}
\newcommand{\first}{\text{FIRST}}
\newcommand{\follow}{\text{FOLLOW}}

\setlength{\columnseprule}{0.5pt}

\title{Assignment 4}

\begin{document}

\maketitle

The grammar \(G\) is given as follows:

\begin{equation*}
    \begin{aligned}
        & (1) & S & \rightarrow \mbfa B, \\
        & (2) & B & \rightarrow S + B, \\
        & (3) & B & \rightarrow \epsilon.
    \end{aligned}
\end{equation*}

Consider the augmented grammar \(G'\):

\begin{equation*}
    \begin{aligned}
        S' & \rightarrow S, \\
        S & \rightarrow \mbfa B, \\
        B & \rightarrow S + B \mid \epsilon.
    \end{aligned}
\end{equation*}

\section{Exercise 1}

\begin{enumerate}
    \item Calculate the FIRST and FOLLOW sets for \(G\).
    \begin{align*}
        \first(S) &= \{ \mbfa \}, \\
        \first(B) &= \{ \mbfa, \epsilon \}, \\
        \follow(S) &= \{ +, \$ \}, \\
        \follow(B) &= \{ +, \$ \}.
    \end{align*}

    Construct the collection of sets of LR(0) items for \(G'\).
    \begin{multicols}{3}
        \begin{enumerate}
            \item[$I_0$:]
            \[
                \begin{aligned}
                    S' & \rightarrow \cdot S \\
                    S & \rightarrow \cdot \mbfa B
                \end{aligned}
            \]
    
            \item[$I_1$:]
            \[
                \begin{aligned}
                    S' & \rightarrow S \cdot
                \end{aligned}
            \]
    
            \item[$I_2$:]
            \[
                \begin{aligned}
                    S & \rightarrow \mbfa \cdot B \\
                    B & \rightarrow \cdot S + B \\
                    B & \rightarrow \cdot \\
                    S & \rightarrow \cdot \mbfa B
                \end{aligned}
            \]

            \item[$I_3$:]
            \[
                \begin{aligned}
                    S & \rightarrow \mbfa B \cdot
                \end{aligned}
            \]
    
            \item[$I_4$:]
            \[
                \begin{aligned}
                    B & \rightarrow S \cdot + B
                \end{aligned}
            \]
    
            \item[$I_5$:]
            \[
                \begin{aligned}
                    B & \rightarrow S + \cdot B \\
                    B & \rightarrow \cdot S + B \\
                    B & \rightarrow \cdot \\
                    S & \rightarrow \cdot \mbfa B
                \end{aligned}
            \]
    
            \item[$I_6$:]
            \[
                \begin{aligned}
                    B & \rightarrow S + B \cdot
                \end{aligned}
            \]
        \end{enumerate}
    \end{multicols}

    Construct the SLR(1) parsing table for \(G'\).
    \begin{center}
        \begin{tabular}{c|ccc|cc}
            \hline \hline
            \multirow{2}{*}{\textsc{State}} & \multicolumn{3}{c|}{\textsc{Action}} & \multicolumn{2}{c}{\textsc{Goto}} \\ \cline{2-6}
                            & \(\mbfa\) & \(+\) & \(\$\) & \(S\) & \(B\) \\ \hline
            0 & s2 & & & 1 & \\
            1 & & & acc & & \\
            2 & s2 & r3 & r3 & 4 & 3 \\
            3 & & r1 & r1 & & \\
            4 & & s5 & & & \\
            5 & s2 & r3 & r3 & 4 & 6 \\
            6 & & r2 & r2 & & \\
            \hline
        \end{tabular}
    \end{center}

    \item Yes, it is.
    
    \item Yes, it can.
    \begin{center}
        \begin{tabular}{r|l|l|r|l}
            \hline \hline
            & \textsc{Stack} & \textsc{Symbols} & \textsc{Input} & \textsc{Action} \\ \hline
            (1) & 0 & & \(\mbfa\mbfa\mbfa\mbfa\)\(+\)\(++\)\$ & shift \\
            (2) & 0 2 & \(\mbfa\) & \(\mbfa\mbfa\mbfa\)\(+\)\(++\)\$ & shift \\
            (3) & 0 2 2 & \(\mbfa\mbfa\) & \(\mbfa\mbfa\)\(+\)\(++\)\$ & shift \\
            (4) & 0 2 2 2 & \(\mbfa\mbfa\mbfa\) & \(\mbfa\)\(+\)\(++\)\$ & shift \\
            (5) & 0 2 2 2 2 & \(\mbfa\mbfa\mbfa\mbfa\) & \(+\)\(++\)\$ & reduce by \(B \rightarrow \epsilon\) \\
            (6) & 0 2 2 2 2 3 & \(\mbfa\mbfa\mbfa\mbfa B\) & \(+\)\(++\)\$ & reduce by \(S \rightarrow \mbfa B\) \\
            (7) & 0 2 2 2 4 & \(\mbfa\mbfa\mbfa S\) & \(+\)\(++\)\$ & shift \\
            (8) & 0 2 2 2 4 5 & \(\mbfa\mbfa\mbfa S+\) & \(++\)\$ & reduce by \(B \rightarrow \epsilon\) \\
            (9) & 0 2 2 2 4 5 6 & \(\mbfa\mbfa\mbfa S+B\) & \(++\)\$ & reduce by \(B \rightarrow S + B\) \\
            (10) & 0 2 2 2 3 & \(\mbfa\mbfa\mbfa B\) & \(++\)\$ & reduce by \(S \rightarrow \mbfa B\) \\
            (11) & 0 2 2 4 & \(\mbfa\mbfa S\) & \(++\)\$ & shift \\
            (12) & 0 2 2 4 5 & \(\mbfa\mbfa S+\) & \(+\)\$ & reduce by \(B \rightarrow \epsilon\) \\
            (13) & 0 2 2 4 5 6 & \(\mbfa\mbfa S+B\) & \(+\)\$ & reduce by \(B \rightarrow S + B\) \\
            (14) & 0 2 2 3 & \(\mbfa\mbfa B\) & \(+\)\$ & reduce by \(S \rightarrow \mbfa B\) \\
            (15) & 0 2 4 & \(\mbfa S\) & \(+\)\$ & shift \\
            (16) & 0 2 4 5 & \(\mbfa S+\) & \$ & reduce by \(B \rightarrow \epsilon\) \\
            (17) & 0 2 4 5 6 & \(\mbfa S+B\) & \$ & reduce by \(B \rightarrow S + B\) \\
            (18) & 0 2 3 & \(\mbfa B\) & \$ & reduce by \(S \rightarrow \mbfa B\) \\
            (19) & 0 1 & \(S\) & \$ & accept \\
            \hline
        \end{tabular}
    \end{center}
\end{enumerate}

\section{Exercise 2}

\begin{enumerate}
    \item Construct the collection of sets of LR(1) items for \(G'\).
    \begin{multicols}{4}
        \begin{enumerate}
            \item[$I_0$:]
            \[
                \begin{aligned}
                    S' & \rightarrow \cdot S,&\$ \\
                    S & \rightarrow \cdot \mbfa B,&\$
                \end{aligned}
            \]
    
            \item[$I_1$:]
            \[
                \begin{aligned}
                    S' & \rightarrow S \cdot,&\$
                \end{aligned}
            \]
    
            \item[$I_2$:]
            \[
                \begin{aligned}
                    S & \rightarrow \mbfa \cdot B,&\$ \\
                    B & \rightarrow \cdot S + B,&\$ \\
                    B & \rightarrow \cdot,&\$ \\
                    S & \rightarrow \cdot \mbfa B,&+
                \end{aligned}
            \]

            \item[$I_3$:]
            \[
                \begin{aligned}
                    S & \rightarrow \mbfa B \cdot,&\$
                \end{aligned}
            \]
    
            \item[$I_4$:]
            \[
                \begin{aligned}
                    B & \rightarrow S \cdot + B,&\$
                \end{aligned}
            \]

            \item[$I_5$:]
            \[
                \begin{aligned}
                    B & \rightarrow S + \cdot B,&\$ \\
                    B & \rightarrow \cdot S + B,&\$ \\
                    B & \rightarrow \cdot,&\$ \\
                    S & \rightarrow \cdot \mbfa B,&+
                \end{aligned}
            \]

            \item[$I_6$:]
            \[
                \begin{aligned}
                    B & \rightarrow S + B \cdot,&\$
                \end{aligned}
            \]

            \item[$I_7$:]
            \[
                \begin{aligned}
                    S & \rightarrow \mbfa \cdot B,&+ \\
                    B & \rightarrow \cdot S + B,&+ \\
                    B & \rightarrow \cdot,&+ \\
                    S & \rightarrow \cdot \mbfa B,&+
                \end{aligned}
            \]

            \item[$I_8$:]
            \[
                \begin{aligned}
                    S & \rightarrow \mbfa B \cdot,&+
                \end{aligned}
            \]

            \item[$I_9$:]
            \[
                \begin{aligned}
                    B & \rightarrow S \cdot + B,&+
                \end{aligned}
            \]

            \item[$I_{10}$:]
            \[
                \begin{aligned}
                    B & \rightarrow S + \cdot B,&+ \\
                    B & \rightarrow \cdot S + B,&+ \\
                    B & \rightarrow \cdot,&+ \\
                    S & \rightarrow \cdot \mbfa B,&+
                \end{aligned}
            \]

            \item[$I_{11}$:]
            \[
                \begin{aligned}
                    B & \rightarrow S + B \cdot,&+
                \end{aligned}
            \]
        \end{enumerate}
    \end{multicols}

    Construct the canonical LR(1) parsing table for \(G'\).
    
    \begin{center}
        \begin{tabular}{c|ccc|cc}
            \hline \hline
            \multirow{2}{*}{\textsc{State}} & \multicolumn{3}{c|}{\textsc{Action}} & \multicolumn{2}{c}{\textsc{Goto}} \\ \cline{2-6}
                            & \(\mbfa\) & \(+\) & \(\$\) & \(S\) & \(B\) \\ \hline
            0 & s2 & & & 1 & \\
            1 & & & acc & & \\
            2 & s7 & & r3 & 4 & 3 \\
            3 & & & r1 & & \\
            4 & & s5 & & & \\
            5 & s7 & & r3 & 4 & 6 \\
            6 & & & r2 & & \\
            7 & s7 & r3 & & 9 & 8 \\
            8 & & r1 & & & \\
            9 & & s10 & & & \\
            10 & s7 & r3 & & 9 & 11 \\
            11 & & r2 & & & \\
            \hline
        \end{tabular}
    \end{center}

    \item Yes, it can.

    \begin{longtable}{r|l|l|r|l}
        \hline \hline
        & \textsc{Stack} & \textsc{Symbols} & \textsc{Input} & \textsc{Action} \\ \hline
        \endfirsthead
        \multicolumn{5}{@{}l}{(continued)} \\
        \hline \hline
        & \textsc{Stack} & \textsc{Symbols} & \textsc{Input} & \textsc{Action} \\ \hline
        \endhead
        \hline
        \endfoot
        (1) & 0 & & \(\mbfa\mbfa\mbfa\mbfa\)\(+\)\(++\)\$ & shift \\
        (2) & 0 2 & \(\mbfa\) & \(\mbfa\mbfa\mbfa\)\(+\)\(++\)\$ & shift \\
        (3) & 0 2 7 & \(\mbfa\mbfa\) & \(\mbfa\mbfa\)\(+\)\(++\)\$ & shift \\
        (4) & 0 2 7 7 & \(\mbfa\mbfa\mbfa\) & \(\mbfa\)\(+\)\(++\)\$ & shift \\
        (5) & 0 2 7 7 7 & \(\mbfa\mbfa\mbfa\mbfa\) & \(+\)\(++\)\$ & reduce by \(B \rightarrow \epsilon\) \\
        (6) & 0 2 7 7 7 8 & \(\mbfa\mbfa\mbfa\mbfa B\) & \(+\)\(++\)\$ & reduce by \(S \rightarrow \mbfa B\) \\
        (7) & 0 2 7 7 9 & \(\mbfa\mbfa\mbfa S\) & \(+\)\(++\)\$ & shift \\
        (8) & 0 2 7 7 9 10 & \(\mbfa\mbfa\mbfa S+\) & \(++\)\$ & reduce by \(B \rightarrow \epsilon\) \\
        (9) & 0 2 7 7 9 10 11 & \(\mbfa\mbfa\mbfa S+B\) & \(++\)\$ & reduce by \(B \rightarrow S + B\) \\
        (10) & 0 2 7 7 8 & \(\mbfa\mbfa\mbfa B\) & \(++\)\$ & reduce by \(S \rightarrow \mbfa B\) \\
        (11) & 0 2 7 9 & \(\mbfa\mbfa S\) & \(++\)\$ & shift \\
        (12) & 0 2 7 9 10 & \(\mbfa\mbfa S+\) & \(+\)\$ & reduce by \(B \rightarrow \epsilon\) \\
        (13) & 0 2 7 9 10 11 & \(\mbfa\mbfa S+B\) & \(+\)\$ & reduce by \(B \rightarrow S + B\) \\
        (14) & 0 2 7 8 & \(\mbfa\mbfa B\) & \(+\)\$ & reduce by \(S \rightarrow \mbfa B\) \\
        (15) & 0 2 9 & \(\mbfa S\) & \(+\)\$ & shift \\
        (16) & 0 2 9 10 & \(\mbfa S+\) & \$ & reduce by \(B \rightarrow \epsilon\) \\
        (17) & 0 2 9 10 11 & \(\mbfa S+B\) & \$ & reduce by \(B \rightarrow S + B\) \\
        (18) & 0 2 8 & \(\mbfa B\) & \$ & reduce by \(S \rightarrow \mbfa B\) \\
        (19) & 0 1 & \(S\) & \$ & accept \\
    \end{longtable}
\end{enumerate}

\section{Exercise 3}

\begin{enumerate}
    \item There are 5 pairs of sets of items that can be merged.
    
    $I_2$ and $I_7$ are replaced by their union:
    \[
        \begin{aligned}
            &I_{2,7}: &S &\rightarrow \mbfa \cdot B,&+/\$ \\
            & &B &\rightarrow \cdot S + B,&+/\$ \\
            & &B &\rightarrow \cdot,&+/\$ \\
            & &S &\rightarrow \cdot \mbfa B,&+/\$ \\
        \end{aligned}
    \]
    
    $I_3$ and $I_8$ are replaced by their union:
    \[
        \begin{aligned}
            &I_{3,8}: &S &\rightarrow \mbfa B \cdot,&+/\$ \\
        \end{aligned}
    \]

    $I_4$ and $I_9$ are replaced by their union:
    \[
        \begin{aligned}
            &I_{4,9}: &B &\rightarrow S \cdot + B,&+/\$ \\
        \end{aligned}
    \]

    $I_5$ and $I_{10}$ are replaced by their union:
    \[
        \begin{aligned}
            &I_{5,10}: &B &\rightarrow S + \cdot B,&+/\$ \\
            & &B &\rightarrow \cdot S + B,&+/\$ \\
            & &B &\rightarrow \cdot,&+/\$ \\
            & &S &\rightarrow \cdot \mbfa B,&+/\$ \\
        \end{aligned}
    \]

    $I_6$ and $I_{11}$ are replaced by their union:
    \[
        \begin{aligned}
            &I_{6,11}: &B &\rightarrow S + B \cdot,&+/\$ \\
        \end{aligned}
    \]

    The LALR(1) parsing table for \(G'\) is as follows.
    
    \begin{center}
        \begin{tabular}{c|ccc|cc}
            \hline \hline
            \multirow{2}{*}{\textsc{State}} & \multicolumn{3}{c|}{\textsc{Action}} & \multicolumn{2}{c}{\textsc{Goto}} \\ \cline{2-6}
                            & \(\mbfa\) & \(+\) & \(\$\) & \(S\) & \(B\) \\ \hline
            0 & s2,7 & & & 1 & \\
            1 & & & acc & & \\
            2,7 & s2,7 & r3 & r3 & 4,9 & 3,8 \\
            3,8 & & r1 & r1 & & \\
            4,9 & & s5,10 & & & \\
            5,10 & s2,7 & r3 & r3 & 4,9 & 6,11 \\
            6,11 & & r2 & r2 & & \\
            \hline
        \end{tabular}
    \end{center}

    \item Yes, it can.
    
    \begin{longtable}{r|l|l|r|l}
        \hline \hline
        & \textsc{Stack} & \textsc{Symbols} & \textsc{Input} & \textsc{Action} \\ \hline
        \endfirsthead
        \multicolumn{5}{@{}l}{(continued)} \\
        \hline \hline
        & \textsc{Stack} & \textsc{Symbols} & \textsc{Input} & \textsc{Action} \\ \hline
        \endhead
        \hline
        \endfoot
        (1) & 0 & & \(\mbfa\mbfa\mbfa\mbfa +++\)\$ & shift \\
        (2) & 0\ \  2,7 & \(\mbfa\) & \(\mbfa\mbfa\mbfa\)\(+\)\(++\)\$ & shift \\
        (3) & 0\ \  2,7\ \  2,7 & \(\mbfa\mbfa\) & \(\mbfa\mbfa\)\(+\)\(++\)\$ & shift \\
        (4) & 0\ \  2,7\ \  2,7\ \  2,7 & \(\mbfa\mbfa\mbfa\) & \(\mbfa\)\(+\)\(++\)\$ & shift \\
        (5) & 0\ \  2,7\ \  2,7\ \  2,7\ \  2,7 & \(\mbfa\mbfa\mbfa\mbfa\) & \(+\)\(++\)\$ & reduce by \(B \rightarrow \epsilon\) \\
        (6) & 0\ \  2,7\ \  2,7\ \  2,7\ \  2,7\ \  3,8 & \(\mbfa\mbfa\mbfa\mbfa B\) & \(+\)\(++\)\$ & reduce by \(S \rightarrow \mbfa B\) \\
        (7) & 0\ \  2,7\ \  2,7\ \  2,7\ \  4,9 & \(\mbfa\mbfa\mbfa S\) & \(+\)\(++\)\$ & shift \\
        (8) & 0\ \  2,7\ \  2,7\ \  2,7\ \  4,9\ \  5,10 & \(\mbfa\mbfa\mbfa S+\) & \(++\)\$ & reduce by \(B \rightarrow \epsilon\) \\
        (9) & 0\ \  2,7\ \  2,7\ \  2,7\ \  4,9\ \  5,10\ \  6,11 & \(\mbfa\mbfa\mbfa S+B\) & \(++\)\$ & reduce by \(B \rightarrow S + B\) \\
        (10) & 0\ \ 2,7\ \  2,7\ \  2,7\ \  3,8 & \(\mbfa\mbfa\mbfa B\) & \(++\)\$ & reduce by \(S \rightarrow \mbfa B\) \\
        (11) & 0\ \ 2,7\ \  2,7\ \  4,9 & \(\mbfa\mbfa S\) & \(++\)\$ & shift \\
        (12) & 0\ \ 2,7\ \  2,7\ \  4,9\ \  5,10 & \(\mbfa\mbfa S+\) & \(+\)\$ & reduce by \(B \rightarrow \epsilon\) \\
        (13) & 0\ \ 2,7\ \  2,7\ \  4,9\ \  5,10\ \  6,11 & \(\mbfa\mbfa S+B\) & \(+\)\$ & reduce by \(B \rightarrow S + B\) \\
        (14) & 0\ \ 2,7\ \  2,7\ \  3,8 & \(\mbfa\mbfa B\) & \(+\)\$ & reduce by \(S \rightarrow \mbfa B\) \\
        (15) & 0\ \ 2,7\ \  4,9 & \(\mbfa S\) & \(+\)\$ & shift \\
        (16) & 0\ \ 2,7\ \  4,9\ \  5,10 & \(\mbfa S+\) & \$ & reduce by \(B \rightarrow \epsilon\) \\
        (17) & 0\ \ 2,7\ \  4,9\ \  5,10\ \  6,11 & \(\mbfa S+B\) & \$ & reduce by \(B \rightarrow S + B\) \\
        (18) & 0\ \ 2,7\ \  3,8 & \(\mbfa B\) & \$ & reduce by \(S \rightarrow \mbfa B\) \\
        (19) & 0\ \ 1 & \(S\) & \$ & accept \\
    \end{longtable}
\end{enumerate}

\end{document}
